\documentclass[a4paper]{scrartcl}
\usepackage[latin9]{inputenc}
\usepackage{a4wide}
\usepackage{amsmath}			%Mathe
\usepackage{amstext}			%Mathe
\usepackage{dsfont}
\parindent0cm
\usepackage{graphicx}			%Importieren von Grafiken
\usepackage[ngerman]{babel}		%f�r deutsches Dokument mit Inhaltsverzeichnis
\usepackage{amssymb}
\usepackage{pdflscape}			%f�r Querformat
\usepackage{subfigure}			%f�r mehrere Bilder nebeneinander
\usepackage{enumerate}

\usepackage{tikz}
\usetikzlibrary{shapes, arrows, positioning}
\usetikzlibrary{backgrounds}


%Globale Trennliste
\usepackage[T1]{fontenc}

%---------------------------------------------------------------------------------------
\begin{document}

\section*{PyMoskito}

PyMoskito steht f�r \textbf{Py}thon based \textbf{Mo}dular \textbf{S}imulation \& Postprocessing \textbf{Ki}ckass \textbf{To}olbox und kann als universelles Werkzeug zur Simulation von typischen Problemen in der Regelungstechnik und anderen Bereichen eingesetzt werden. Die st�rken liegen bei der Simulation von vielen Konfigurationen mit diversen Variationen in der Struktur oder in den Parametern.


Der Grundstein f�r PyMoskito wurde im Wintersemester 2014/2015 in Form eines Oberseminars zum Thema "`Untersuchung der Ball and Beam Problematik"' gelegt. Dieses Oberseminar beinhaltete unter anderem die Aufgabe das System mit Python zu simulieren. Daraus entwickelte sich St�ck f�r St�ck die hier vorliegende Toolbox.

% aufheben f�r Getting Started
% Python 2.7
%PyMoskito ist als Paket konzipiert und kann durch die setup.py installiert werden. Weiterhin sind Abh�ngigkeiten zu anderen Paketen zu nennen, die f�r PyMoskito ben�tigt.
%\begin{itemize}
%	\item numpy
%	\item sympy
%	\item scipy
%	\item PyQt4
%	\item cPickle
%	\item vtk
%\end{itemize}

Zur �bersicht und einfachen Bedienung der Simulation enth�lt PyMoskito eine Benutzeroberfl�che (GUI). Mit Hilfe der GUI werden folgende Funktionen abgedeckt:
\begin{itemize}
	\item einfache Konfiguration des Regelkreises
	\item Simulation einer eingestellten Regelkreisstruktur
	\item Laden von gespeicherten Simulationskonfigurationen
	\item automatische Simulation von Simulationss�tzen
	\item Speichern von Simulationsergebnissen zur sp�teren Bearbeitung
	\item Schnittstelle zur 3D-Visualisierung des physikalischen Systems mit Hilfe von VTK
	\item Playback Funktion f�r die 3D-Visualisierung
	\item Darstellung aller Simulationsergebnisse direkt und ohne Umwege
	\item M�glichkeit zur individuellen Weiterbearbeitung von Simulationsergebnissen durch Postprozessoren
\end{itemize}

PyMoskito bietet einen Simulationskern zur Ausf�hrung der in Abbildung \ref{abb:schema_des_systemmodells} dargestellten Regelkreisstruktur und eine Nachbearbeitung von Simulationsergebnissen an. Die Module dieses Regelkreises sind hierbei einzeln konfigurierbar und bieten so die M�glichkeit verschiedenste Szenarien gleichzeitig abzudecken. Sie unterteilen sich dabei in zwei verschiedene Gruppen:
\begin{enumerate}[I.]
	\item generische, daher vom konkreten System unabh�ngige Bl�cke die damit auch f�r
	andere Systeme Verwendung finden k�nnen und
	\item systemabh�ngige, speziell zugeschnittene Bl�cke, die f�r jedes System neu implementiert werden m�ssen.
\end{enumerate}

\begin{figure}
	\begin{center}
		\tikzstyle{block} = [draw, rectangle, fill=blue!00, minimum height=3em, minimum width=6em]
		\tikzstyle{systemBlock} = [draw, rectangle, fill=blue!20, minimum height=3em, minimum width=6em]
		\tikzstyle{sum} = [draw, circle, node distance=10mm]
		\tikzstyle{input} = [coordinate]
		\tikzstyle{output} = [coordinate]
		\tikzstyle{pinstyle} = [pin edge={to-,thin,black}]
		
		\begin{tikzpicture}[auto, node distance=10mm and 10mm,>=latex']
			% place blocks
			\node [block]			(trajektorie)	{Trajektorie};
			\node [systemBlock, right=of trajektorie]	(regler)	{Regler};
			\node [systemBlock, above=of regler]	(vorsteuerung)	{Vorsteuerung};
			\node [sum, right=of regler]	(summierer2)	{};
			\node [block, right=of summierer2]	(limitierung)	{Limitierung};
			\node [systemBlock, right=of limitierung]	(system)	{System};
			\node [output, right=of system]	(ausgang) 	{};
			\node [block, below=of system, pin={[pinstyle]below:$d$}]	(sensor)	{Sensor};
			\node [input, below=of sensor]	(stoerung) 	{};
			\node [systemBlock, left=of sensor]	(beobachter)	{Beobachter};
			
			% draw connections
			\draw [->]	(trajektorie)	-- node [name=w, below] {$w$}	(regler);
			\draw [->]	(w)		|- (vorsteuerung);
			\draw [->]	(regler)	-- node {$r$}	(summierer2);
			\draw [->]	(vorsteuerung.east) -| node [near end] {$v$}	(summierer2);
			\draw [->]	(summierer2)	-- node {$u$}	(limitierung);
			\draw [->]	(limitierung)	-- node {$l$}	(system);
			\draw [->]	(system)	-- node [name=y] {$y$} (ausgang);
			\draw [->]	(y) |- (sensor);
			\draw [->]	(sensor)	-- node {$s$}	(beobachter);
			\draw [->]	(beobachter)	-| node[near end] {$\hat{x}$}	(regler);
			
			% draw branches
			\fill (y|-system) circle [radius=1.5pt];
			\fill (w|-trajektorie) circle [radius=1.5pt];
		\end{tikzpicture}
		
		\caption[Schema des Regelkreises]{Schematische Darstellung des Regelkreises. Blau hinterlegte Komponenten
			sind an das jeweilige System anzupassen, alle anderen
			Bl�cke sind generisch verwendbar.}
		\label{abb:schema_des_systemmodells}
	\end{center}
\end{figure}

Das System ist Modular aufgebaut, sodass Bl�cke ausgetauscht oder weggelassen werden k�nnen. Neben dem eigentlichen Simulationssteuerung werden generische Klassen zur Verf�gung gestellt, welche h�ufig benutzte Simulationselemente enthalten. Dazu z�hlen:
\begin{itemize}
	\item ODEInt, ein Integrator zur L�sung von Differentialgleichungssystemen 1. Ordnung
	\item SmoothTransition, ein Trajektoriengenerator zur Erzeugung von glatten �berg�ngen zwischen gegebenen Zust�nden und Zeitpunkten
	\item HarmonicTrajectory, ein Trajektoriengenerator zur Erzeugung von harmonischen Sollwerten mit gegebener Frequenz und Amplitude
	\item PIDController, ein Standard Regler in der Regelungstechnik
	% sp�ter eventuell Implementierung als Basisklasse um dann von dieser abzuleiten (in Konflikt mit blauer Box)
%	\item Limitierung (noch zu implementieren)
%	\item Sensor (noch zu implementieren)
\end{itemize}

PyMoskito kommt mit einer Vielzahl von bekannten Beispielen aus der Regelunsgtechnik die als Anleitung und Hilfe bei der Implementierung eigener Problemstellung dienen. Ein Beispiel:

Es handelt sich um das in der Regelungstechnik beliebte "`Ball and Beam"' System. Bei der "`Ball and Beam"' Problematik geht es darum, einen Ball auf einem in der Mitte drehbar gelagerten Balken auf eine bestimmte Position zu regeln. Aus diesem Beispiel ist auch ersichtlich, welche Module bzw. Dateien bei einer neuen Problemstellung zu erstellen sind. Hierf�r existieren in PyMoskito verschiedene Basisklassen, auf die aufgebaut werden kann. 
% Dies stellt sicher das alle f�r die Simulation ben�tigten Bestandteile implementiert werden. 
Die folgende Aufstellung zeigt was konkret zu tun ist:
\begin{itemize}
	\item Implementierung der Systemdynamik als Zustandsraummodell (right hand side)
	\item in der auszuf�hrenden Datei muss die Benutzeroberfl�che erstellt und initialisiert werden
	\item Regelgesetze zur Stabilisierung des Regelkreises
	\item Entwurf einer Standardsimulationskonfiguration
	\item Optional: 3D-Visualisierung mit Hilfe von VTK
	%	\item main.py, beinhaltet die Erstellung der grafischen Oberfl�che
	%	\item visualizer_mpl.py, dient zur VTK basierten Visualisierung
	%	\item settings.py, enth�lt die Parameter und Konstanten des Systems
	%	\item model.py, hier wird das Modell als Zustandsraummodell implementiert
	%	\item controll.py, dient zur Implementierung von eigenen Reglergesetzen
	%	\item default.sreg, kann zur Konfiguration einer Simulation benutzt werden
\end{itemize}

\end{document}
